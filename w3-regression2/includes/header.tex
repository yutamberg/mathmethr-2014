%%% Работа с русским языком
\usepackage[english,russian]{babel}   %% загружает пакет многоязыковой вёрстки
\usepackage{fontspec}      %% подготавливает загрузку шрифтов Open Type, True Type и др.
\defaultfontfeatures{Ligatures={TeX},Renderer=Basic,Scale=0.75}  %% свойства шрифтов по умолчанию
\setmainfont[Ligatures={TeX,Historic}]{Times New Roman} %% задаёт основной шрифт документа
\setsansfont[Ligatures={TeX,Historic}]{Verdana} %% задаёт шрифт без засечек
\setmonofont{CMU Typewriter Text}
% \setmainfont[Ligatures={TeX,Historic}]{CMU Serif} %% задаёт основной шрифт документа
% \setsansfont{CMU Sans Serif}                    %% задаёт шрифт без засечек
% \setmonofont{CMU Typewriter Text}               %% задаёт моноширинный шрифт

\linespread{0.65}

\definecolor{links}{HTML}{2A1B81}
\hypersetup{colorlinks=true,linkcolor=,urlcolor=links,pdfview=FitH,pdfpagelayout=SinglePage, unicode=true,breaklinks=true}

\usepackage{hyperref}
\usepackage[labelformat=empty]{caption}

\usepackage{enumerate} % for customization of enumerated lists
\usepackage{subcaption} % for subfigures
%\usepackage{graphicx} %automatically included by beamer

\usepackage{amsmath}
\usepackage{amssymb}

%\addto{\captionsrussian}{\renewcommand*{\figurename}{рис.}}

\usepackage{array}
\renewcommand{\arraystretch}{2}
\newcolumntype{L}[1]{>{\raggedright\let\newline\\\arraybackslash\hspace{0pt}}m{#1}}
\newcolumntype{C}[1]{>{\centering\let\newline\\\arraybackslash\hspace{0pt}}m{#1}}
\newcolumntype{R}[1]{>{\raggedleft\let\newline\\\arraybackslash\hspace{0pt}}m{#1}}


